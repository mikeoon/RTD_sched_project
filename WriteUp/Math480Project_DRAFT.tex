\documentclass[12pt]{article}

\oddsidemargin=.05in
\evensidemargin=.05in
\topmargin=-0.5in
\textwidth=6.5in
\textheight=9in
%\pagestyle{empty}

\usepackage{multicol}
\usepackage{amsmath,amssymb,amsthm}
\usepackage{easylist}
\usepackage{graphicx}
\newtheorem*{theorem}{Theorem}
\newtheorem*{claim}{Claim}
\newtheorem*{proposition}{Proposition}
\renewcommand{\qedsymbol}{\ensuremath{\blacksquare}}
\usepackage{fancyhdr}
\pagestyle{fancy}
\fancyhf{}
\lhead{Math 480 Project}
%\chead{Captain Scheduling for Ride the Ducks}
\rhead{Bickel and Yoon}
\cfoot{\thepage}


\begin{document}

\title{Math 480 Project Draft \\ Captain Scheduling for Ride the Ducks} 

\author{Zach Bickel and Michael Yoon}
\date{March 18, 2015}
\maketitle

\begin{abstract}
This paper describes the model and solution we derived from a scheduling problem for a 
small business, Ride the Ducks of Seattle. The objective was to create scheduling 
software that not only created a schedule, but also satisfied the wants and needs of the 
company as well as the workers. The problem is modeled as a Mixed Integer Problem 
and fitted with a branch and bound algorithm. After simplifying the problem at hand to 
fix the model we wanted to use, we were able to derive decision variables and a cost 
function to minimize. We used a two-step process. First, we began with a schedule 
creator with no branch and bound algorithm implemented simply to satisfy the constraints of the 
workers and the company. We then implemented the 
branch and bound algorithm to obtain a feasible solution with a better score than the original solution. Our finished product will be used 
by Ride the Ducks with the option of changing constraints to fit their week-to-week 
schedule making.
\end{abstract}


\section*{Background}
Scheduling problems are very common for small businesses where many managers are still doing it by hand. Management at Ride the Ducks of Seattle currently spends upwards of ten hours per week scheduling captains for tours. Our goal is to build scheduling software that will drastically cut this time down and automatically build a schedule. This will be flexible enough so that a user will be able to update the constraints as they change. 

Ride the Ducks of Seattle offers continuous tours of downtown Seattle every day of the year except for Christmas, Christmas Eve and Thanksgiving. The Land \& Water Tour is a 90-minute tour that begins from two separate locations in downtown Seattle. Each tour is directed by a Captain, who operates an amphibious vehicle called a DUKW, also known as a duck. After each 90-minute tour is finished, there is a 30-minute break period for de-boarding the current tour, boarding the next tour, and a small break for the Captain. Each Captain will usually work 5 tours a day, 4 days a week, although there is a lot of variability from this in practice. Throughout this paper, we refer to 2-hour tour blocks, or simply 2-hour blocks, to denote a 90-minute tour followed by a 30-minute break period.

Scheduling the Captains in a way that satisfies the company as well as the 
individual Captains can be tricky. There are individual captain constraints that we have to deal with while filling the tours Ride the Ducks wants to run. Captain constraints can vary week to week and the amount of tours run in a day depends seasonally with the warmer seasons needing more tours than the cooler seasons. Due to the need of a flexible scheduling system, the notion of making the scheduling software easy to edit is crucial. The company can then 
make their own schedules with changing data. 

\section*{Goal}
Out goal is to automate a scheduling system for Ride the Ducks Captains to cover all shifts while satisfying each Captain's individual constraints. % and attempting to minimize 4-day stretches.

\section*{Simplifications}
Ride the Ducks is open 7 days a week and generally runs tours from 9:20 AM to 7:20 PM during the busy season. To make the software future-proof, we extend the schedule so that the first tours can start as early as 9 AM and the last tours can start as late as 9 PM. There is one captain 
and one duck per tour. Each captain is not supposed to work more than four days a week 
and there are special constraints for individual captains, such as “Captain A cannot work 
on Tuesday”. Our objective is to minimize the total number of 4-day stretches all captains 
have to work. Minimizing these 4-day stretches is a way to keep the workers rested and 
keep hours even among the captains that need them.  We started by figuring out what a 
Ride the Ducks schedule looks like. The schedule is broken down into which captains are
working what days and the times that day that captain will be running a tour. The 
schedule also gives one or two captains that are “on-call” so that if a captain has a 
conflict day of or more tours than expected or scheduled the captains “on-call” can come 
in and help. From the schedule given by the company we were able to break down the 
schedule into decisions. The decisions include, how many captains are available, how 
many captains we want in a day, how many time slots for tours, and a captain’s own 
restrictions on which days he or she can work. These decisions we believe are the most 
important to creating a schedule. With these decisions we made assumptions to certain 
aspects of the variables involved. For the tours, we capped the total length at 120 minutes 
which incudes the tour and rest time for the captains. The time slots for those tours are set 
to a default of one tour every 10 minutes. The number of captains in our problem is 40, 
with 22 captains scheduled a day, 2 of which are on-call. We simplified each block of 
tours to be 6 blocks of tours per day. A block of tours is the 2-hour period in which a 
captain is out on a tour. So, if the company is open for 12 hours a day, there are 6 blocks 
of tours.

\section*{Literature Overview}
Before we tried to model the problem at hand, we read research papers on
modeling problems similar to our own. One paper we came across discussed America 
West and the Arizona State University using a mathematical model to decide on the best 
airplane boarding strategy (Hogg, 2002). This paper gave us insight on handling mixed integer 
problems and how they are modeled. We were able to better derive the decision variables 
in our own problem after learning about how the ASU and America West team simplified 
their problem. The ASU and America West team simplified the problems of boarding an 
airplane by classifying the problems into two groups (Hogg). This helped us to 
understand how to simplify our own constraints as well as which decision variable we 
wanted to minimize.
We came across Patrick Perkin’s paper on creating weekly timetables for the 
Math Study Center at the University of Washington (Perkins, 2004). Perkin’s paper tackled 
their scheduling problem in a two step process. The first step of their modeling process 
utilized a relaxed version of constraints to just obtain a valid schedule. The second step 
involved stricter constraints that influenced the objective function. This second step 
would go on to produce better schedules. Perkin’s states that these better schedules are 
better optimized and thus produce more efficient schedules, ~7% better (Perkins). 
Another paper we read was by Nicholas Beaumont on scheduling staff that drove 
to and serviced customers (Beaumont, 1997). In Beaumont’s case study, the staff he was trying 
to schedule had many similarities to the staff we were trying to schedule. This included 
constraints on the staff based on times they could work and the demand of their work 
(Beaumont). Using binary values and other variables assigned to certain staff members to 
restrict their start times was something that we wanted to implement in our model. These 
constraints may change by week for our own partner, thus knowing the formulation of 
these constraints was necessary to write up code that would create these constraints for 
our partner to utilize once we are done with the project. Similar to this, the demand of the 
staff in Beaumont’s study and our own problem of varying numbers of tours based on the 
time of year. 
These papers offered us guidance on our own problem and model. Simplify the 
process; solve the problem two ways, one more relaxed and the other fit to tighter 
constraints, and how to model certain restrictions that may change based on the situation. 
All of the ideas listed we utilized to give our partners the best possible solution.

\section*{Decision Variables and Parameters} First, let's define some parameters:
\begin{align*}
n &\equiv \text{Number of captains,}\\
m &\equiv \text{Number of tour slots per 2-hour blocks in one day,}\\
b &\equiv \text{Number of 2-hour blocks in one day,}\\
k &\equiv \text{Number of time slots per day.}
\end{align*}
If there are $n$ captains and $k$ time slots in a day, then there are $7kn$ decision variables. Note that $k = mb$. Our decision variables will represent when the captains will start their shifts. We can understand them as a matrix $X$, where $x_{ijd}$ = 1 if captain $i \in \{1, 2, \dots, n\}$ begins their shift at time $j \in \{1, 2, \dots k\}$ on day $d \in \{1, 2, \dots, 7\}$ and 0 otherwise.

\section*{Objective Function}
Our objective function will represent the number of 4-day stretches captains have to work and our goal will be to minimize this in order to keep captain fatigue low. For readability purposes, let's define a separate function $w(i, d)$ to represent captain $i$ working on day $d$:
$$w(i,d) = \sum_{j = 1}^{k}{x_{ijd}}.$$
So, $w(i,d) = 1$ if captain $i$ works on day $d$ and 0 otherwise. Then $w(i, 1)w(i, 2)w(i, 3)w(i, 4) = 1$ implies that captain works on days 1, 2, 3 and 4. Our objective function $f$ is thus:
$$f(i) = [w(i, 1) \cdots w(i, 4)] + [w(i, 2) \cdots w(i, 5)] + \cdots + [w(i, 4) \cdots w(i, 7)] + [w(i, 5) w(i, 6) w(i, 7) w(i,1)].$$
Note that the last term in the function wraps around to make sure that we are not loading the schedule too heavily towards the weekend shifts.

\section*{Constraints}
\begin{enumerate}
\item[(1)] Number of daily captains constraint: We want to have $c_d$ captains at work on day $d$.
$$\sum_{i = 1}^{n}\sum_{j = 1}^{k}{x_{ijd}} = c_d.$$
\item[(2)] Multiple start times constraint: The captain can only have one start time for each day. 
For each day $d$, and for each captain $i$,only 
$$\sum_{j = 1}^{k}{x_{ijd}} \le 1.$$
\item[(3)] Number of days worked per week constraint: Each captain can work a maximum of 4 days per week. Some have to work less, so we'll define the maximum number of days captain $i$ can work as $y_i$.
For each captain $i$,
$$\sum_{d = 1}^{7}\sum_{j = 1}^{k}{x_{ijd}} \le y_i.$$
\item[(4)] Tours per time slot constraint: We would like to have a certain number of captains available at each time slot in order to run the required number of tours. If we want $t_{jd}$ tours at time slot $j$ on day $d$ we have to first obtain the number of captains that are available for a tour (i.e. haven't run 5 or more tours in a day already). Define captain $i$ starting a shift or eligible for the tour $j$ on day $d$ by $z_{ijd}$.
$$\sum_{i = 1}^{n}{z_{ijd}} = t_{jd}.$$
\item[(5)] Captain unavailability constraint: If captain $i$ is unavailable on day $d$,
$$\sum_{d = 1}^{7}\sum_{j = 1}^{k}{x_{ijd}} = 0.$$
\item[(6)] Captain required tour constraint: Some captains are requested to run specific tours, so if captain $i$ has to work 
day $d$ on time slot $j$,
$$x_{ijd} = 1.$$
\end{enumerate}

\section*{Results} As of this writing, the code has been fairly buggy and we haven't been able to obtain a feasible solution. Sample output from our code is below and it is formatted to show the starting schedule visually. 
\begin{center}
\includegraphics[scale=0.1]{"RTD_result_bad".png}
\end{center}
What this shows is that the captains are front-loaded on each day and none of the tours per time slot constrains are satisfied. Further improvements of our algorithm are needed to obtain a solution. 

\end{document}