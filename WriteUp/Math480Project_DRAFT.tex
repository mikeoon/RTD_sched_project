\documentclass[12pt]{article}

\oddsidemargin=.05in
\evensidemargin=.05in
\topmargin=-0.5in
\textwidth=6.5in
\textheight=9in
%\pagestyle{empty}

\usepackage{amsmath,amssymb,amsthm}
\usepackage{easylist}
\usepackage{graphicx}
\usepackage{tikz}
\usetikzlibrary{arrows,automata}

\begin{document}
\noindent
Zach Bickel\\
Michael Yoon\\
March 2, 2015
\begin{center}
\huge{Math 480 Project Draft}\\
\large{Captain Scheduling for Ride the Ducks}
\end{center}

\begin{abstract}
This paper describes the model and solution we derived from a scheduling problem for a 
small business, Ride the Ducks of Seattle. The objective was to create scheduling 
software that not only created a schedule, but also satisfied the wants and needs of the 
company as well as the workers. The problem is modeled as a Mixed Integer Problem 
and fitted with a branch and bound algorithm. After simplifying the problem at hand to 
fix the model we wanted to use, we were able to derive decision variables and a cost 
function to minimize. We took a two-step process. First, we began with a schedule 
creator with no branch and bound algorithm implemented to satisfy the constraints of the 
workers and the company and create the most basic schedule. We then implemented the 
branch and bound algorithm to minimize the number of captains a day while still 
maintaining the constraints set in place in the first step. Our finished product will be used 
by Ride the Ducks with the option of changing constraints to fit their week-to-week 
schedule making.
\end{abstract}

\section*{Goal}
Scheduling problems are very common for small businesses where many managers are
still doing it by hand. This can lead to hours of tinkering with the schedule to get the most 
out of the workers. Ride the Ducks of Seattle is a major tourist attraction that takes their 
customers on a tour of Seattle on land and water. A captain directs each tour; the captains 
are the ones who operate the vehicles, known as ducks, and gives the tours. The ducks are 
the pride and joy of the company. These vehicles are capable of going on land to water 
and vise versa. Each tour is 2 hours long, which includes about 90 minutes of touring 
with a 30-minute break for the captains. The amount of tours run depends seasonally with 
the warmer seasons needing more tours than the cooler seasons. Scheduling the captains 
and their ducks in a way that satisfies working conditions by the company as well as the 
individual captains can be tricky. Management at Ride the Ducks of Seattle currently 
spends upwards of ten hours per week scheduling captains for tours. Ten hours creating a 
schedule is much too long, we believe we can not only create a schedule for Ride the 
Ducks much faster, but also create a schedule that allows the company to run tours 
efficiently while satisfying the captains’ and the company’s needs. Our goal is to build 
scheduling software that will drastically cut time down and automatically build a 
schedule. This will be flexible enough so that a user will be able to update the constraints
as they come up. Ride the Ducks of Seattle is a very private about the details for how 
their company is run. They keep much of their data to themselves to protect the company 
from those that want them gone. Due to their privacy, the notion of making the 
scheduling software easy to edit and make changes to is crucial. The company can then 
make their own schedules with the data that they might not have given us. With this 
schedule creator, we hope that Ride the Ducks will spend less time working on 
scheduling the captains and spend their newly acquired time efficiently. 

\section*{Simplifications}
Ride the Ducks is open 7 days a week from 9 AM to 9 PM and offers 90-minute tours 
where the driver ideally gets 30 minutes of rest before the next tour. There is one captain 
and one duck per tour. Each captain is not supposed to work more than four days a week 
and there are special constraints for individual captains, such as “Captain A cannot work 
on Tuesday”. Our objective is to minimize the total number of 4-day stretches all captains 
have to work. Minimizing these 4-day stretches is a way to keep the workers rested and 
keep hours even among the captains that need them.  We started by figuring out what a 
Ride the Ducks schedule looks like. The schedule is broken down into which captains are
working what days and the times that day that captain will be running a tour. The 
schedule also gives one or two captains that are “on-call” so that if a captain has a 
conflict day of or more tours than expected or scheduled the captains “on-call” can come 
in and help. From the schedule given by the company we were able to break down the 
schedule into decisions. The decisions include, how many captains are available, how 
many captains we want in a day, how many time slots for tours, and a captain’s own 
restrictions on which days he or she can work. These decisions we believe are the most 
important to creating a schedule. With these decisions we made assumptions to certain 
aspects of the variables involved. For the tours, we capped the total length at 120 minutes 
which incudes the tour and rest time for the captains. The time slots for those tours are set 
to a default of one tour every 10 minutes. The number of captains in our problem is 40, 
with 22 captains scheduled a day, 2 of which are on-call. We simplified each block of 
tours to be 6 blocks of tours per day. A block of tours is the 2-hour period in which a 
captain is out on a tour. So, if the company is open for 12 hours a day, there are 6 blocks 
of tours.


\section*{Decision Variables and Parameters} First, let's define some parameters:
\begin{align*}
n &\equiv \text{Number of captains,}\\
m &\equiv \text{Number of tour slots per 2-hour blocks in one day,}\\
b &\equiv \text{Number of 2-hour blocks in one day,}\\
k &\equiv \text{Number of time slots per day.}
\end{align*}
If there are $n$ captains and $k$ time slots in a day, then there are $7kn$ decision variables. Note that $k = mb$. Our decision variables will represent when the captains will start their shifts. We can understand them as a matrix $X$, where $x_{ijd}$ = 1 if captain $i \in \{1, 2, \dots, n\}$ begins their shift at time $j \in \{1, 2, \dots k\}$ on day $d \in \{1, 2, \dots, 7\}$ and 0 otherwise.

\section*{Objective Function}
Our objective function will represent the number of 4-day stretches captains have to work and our goal will be to minimize this in order to keep captain fatigue low. For readability purposes, let's define a separate function $w(i, d)$ to represent captain $i$ working on day $d$:
$$w(i,d) = \sum_{j = 1}^{k}{x_{ijd}}.$$
So, $w(i,d) = 1$ if captain $i$ works on day $d$ and 0 otherwise. Then $w(i, 1)w(i, 2)w(i, 3)w(i, 4) = 1$ implies that captain works on days 1, 2, 3 and 4. Our objective function $f$ is thus:
$$f(i) = [w(i, 1) \cdots w(i, 4)] + [w(i, 2) \cdots w(i, 5)] + \cdots + [w(i, 4) \cdots w(i, 7)] + [w(i, 5) w(i, 6) w(i, 7) w(i,1)].$$
Note that the last term in the function wraps around to make sure that we are not loading the schedule too heavily towards the weekend shifts.

\section*{Constraints}
\begin{enumerate}
\item[(1)] Number of daily captains constraint: We want to have $c_d$ captains at work on day $d$.
$$\sum_{i = 1}^{n}\sum_{j = 1}^{k}{\mathbf{x}_{ijd}} = c_d$$
\item[(2)] Multiple start times constraint. The captain can only have one start time for each day.
On day $d = 1,\dots, 7$, and for all captains $i = 1,\dots, n$:
$$\sum_{j = k(d-1) + 1}^{kd}{\mathbf{x}_{ij}} \le 1$$
\item[(3)] Number of days worked constraint. Each captain can only work a maximum of 4 days per week. 
For each captain $i$:
$$\sum_{j = 1}^{7k}{\mathbf{x}_{ij}} \le 4$$
\item[(4)] Tours per time slot constraint. We would like to have a minimum number of captains available at each time slot in order to run the required number of tours. If we want $t_j$ tours at time slot $j$:
$$\sum_{i = 1}^{n}{\mathbf{x}_{ij}} \ge t_j$$
\item[(5)] Captain unavailability constraint. If captain $i$ is unavailable on day $d$:
$$\sum_{j = k(d-1) + 1}^{kd}{\mathbf{x}_{ij}} = 0$$
\end{enumerate}

\end{document}